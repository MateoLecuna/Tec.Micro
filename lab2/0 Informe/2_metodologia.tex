\newpage

\section{Metodología}

\subsection{Materiales a utilizar}
\begin{itemize}
    \item Microcontrolador ATmega328P (plataforma Arduino UNO).
    \item 12 pulsadores para las notas de la escala cromática.
    \item 2 pulsadores adicionales para el control de octavas.
    \item Buzzer piezoeléctrico.
    \item Resistencias pull-up internas configuradas por software.
    \item Conexiones con jumpers y protoboard.
\end{itemize}

\subsection{Plotter}

\subsection{Seleccionador de colores}

\subsection{Piano}
\subsubsection{Diseño del sistema}
El sistema se estructuró en torno a los siguientes bloques principales:
\begin{enumerate}
    \item \textbf{Lectura de pulsadores:} Se configuraron los pines digitales como entradas con resistencias pull-up internas. El estado lógico bajo indica que la tecla fue presionada.
    \item \textbf{Generación de notas:} Utilizando los temporizadores del ATmega328P en modo PWM, se programaron las frecuencias correspondientes a cada nota musical. Las notas se definieron en una tabla para facilitar su acceso en el programa.
    \item \textbf{Cambio de octava:} Se reservaron dos pulsadores adicionales para aumentar o disminuir la octava activa. Esto permite variar la frecuencia base de todas las notas según la octava seleccionada.
    \item \textbf{Reproducción de canciones:} La comunicación UART se estableció para recibir comandos externos y activar la ejecución de melodías predefinidas. Estos comandos permiten conmutar entre modo piano manual y modo automático.
\end{enumerate}

\subsubsection{Estado de avance}
Hasta el momento, se han definido los pines correspondientes a las 12 notas musicales y los dos pulsadores de control de octavas. También se determinó la utilización de resistencias pull-up internas para simplificar el diseño del circuito. El siguiente paso consiste en implementar la generación de tonos mediante PWM y la integración de la comunicación serial para las canciones predefinidas.

\subsection{Cerradura electrónica}
