\newpage

\section{Metodología}

\subsection{Materiales a utilizar}
\begin{itemize}
    \item Microcontrolador ATmega328P (plataforma Arduino UNO).
    \item 12 pulsadores para las notas de la escala cromática.
    \item 2 pulsadores adicionales para el control de octavas.
    \item Buzzer piezoeléctrico.
    \item Resistencias pull-up internas configuradas por software.
    \item Conexiones con jumpers y protoboard.
\end{itemize}

\subsection{Plotter}

\subsection{Seleccionador de colores}

\subsection{Piano}
\subsubsection{Diseño del sistema}
El sistema se estructuró en torno a los siguientes bloques principales:
\begin{enumerate}
    \item \textbf{Lectura de pulsadores:} Se configuraron los pines digitales como entradas con resistencias pull-up internas. El estado lógico bajo indica que la tecla fue presionada.
    \item \textbf{Generación de notas:} Utilizando los temporizadores del ATmega328P en modo PWM, se programaron las frecuencias correspondientes a cada nota musical. Las notas se definieron en una tabla para facilitar su acceso en el programa.
    \item \textbf{Cambio de octava:} Se reservaron dos pulsadores adicionales para aumentar o disminuir la octava activa. Esto permite variar la frecuencia base de todas las notas según la octava seleccionada.
    \item \textbf{Reproducción de canciones:} La comunicación UART se estableció para recibir comandos externos y activar la ejecución de melodías predefinidas. Estos comandos permiten conmutar entre modo piano manual y modo automático.
\end{enumerate}

\subsubsection{Obtención y transcripción de melodías}

Para la programación de las canciones predefinidas, fue necesario obtener previamente las secuencias de notas, duraciones y octavas de cada una. 

La primera melodía, denominada \textit{“Asesina”}, fue extraída a partir del video de YouTube \cite{youtube_asesina}. Para identificar las notas y sus duraciones se utilizó una aplicación de piano roll disponible en App Store, con la cual se reprodujo la melodía y se registraron manualmente las notas en una hoja de referencia, indicando la octava correspondiente y el tiempo de ejecución de cada una. Posteriormente, estos valores fueron transferidos al código fuente en forma de arreglos.

La segunda melodía, correspondiente al tema principal de \textit{Super Mario Bros}, se obtuvo a partir de la partitura disponible en el portal MuseScore \cite{musescore_mario}. Dicha fuente incluía además una simulación visual en piano roll, lo que facilitó la identificación de las notas y su duración sin necesidad de realizar el reconocimiento auditivo de las notas, como se hizo con la primera canción. 

En el documento de evidencias anexo se incluyen las hojas manuscritas utilizadas en la transcripción de ambas canciones, junto con una tabla que relaciona las notas musicales con su índice numérico dentro del arreglo de notas del programa. No fue posible incluir el archivo PDF con la partitura de la canción de \textit{Super Mario Bros} debido a que requiere una suscripción paga para su descarga, pero, sin embargo, accediendo al enlace indicado en la bibliografía es posible visualizar la partitura y la simulación en piano roll mencionada anteriormente.

\subsection{Cerradura electrónica}
