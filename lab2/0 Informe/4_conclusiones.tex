
\section{Conclusiones}

\subsection{Plotter}

\subsection{Seleccionador de colores}

\subsection{Piano}
El desarrollo del piano electrónico permitió aplicar conceptos fundamentales del manejo de periféricos del microcontrolador ATmega328P, en particular el uso de temporizadores para la generación de señales PWM y la comunicación UART para el intercambio de comandos externos.  

El sistema demostró un funcionamiento estable y confiable, logrando la reproducción precisa de las 12 notas musicales y una correcta interpretación de las canciones predefinidas. La calidad del sonido obtenida mediante el buzzer fue adecuada, con una frecuencia de salida clara y sin distorsiones perceptibles.  

La implementación de la comunicación serial aportó una capa adicional de control, permitiendo al usuario seleccionar canciones o volver al modo manual de manera sencilla. Si bien no se desarrolló el comando \texttt{STOP} ni el cambio de octavas, los objetivos principales del ejercicio fueron alcanzados con éxito.  

En términos generales, este ejercicio evidenció una correcta integración entre hardware y software, así como una adecuada gestión de temporización y respuesta a eventos. Como líneas de mejora, se puede proponer incorporar un sistema de control de volumen o modulación, implementar el comando de detención, y explorar el uso de interrupciones para optimizar la detección de teclas y la eficiencia del sistema.

Este módulo representa un ejemplo claro de la integración entre teoría de control, electrónica digital y fundamentos acústicos aplicados al desarrollo de sistemas embebidos musicales.

\subsection{Cerradura electrónica}
