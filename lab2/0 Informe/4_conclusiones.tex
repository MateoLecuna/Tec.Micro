
\section{Conclusiones}

\subsection{Plotter}

\subsection{Seleccionador de colores}

\subsection{Piano}
El desarrollo del piano electrónico permitió aplicar conceptos fundamentales del manejo de periféricos del microcontrolador ATmega328P, en particular el uso de temporizadores para la generación de señales PWM y la comunicación UART para el intercambio de comandos externos.  

El sistema demostró un funcionamiento estable y confiable, logrando la reproducción precisa de las 12 notas musicales y una correcta interpretación de las canciones predefinidas. La calidad del sonido obtenida mediante el buzzer fue adecuada, con una frecuencia de salida clara y sin distorsiones perceptibles.  

La implementación de la comunicación serial aportó una capa adicional de control, permitiendo al usuario seleccionar canciones o volver al modo manual de manera sencilla. Si bien no se desarrolló el comando \texttt{STOP} ni el cambio de octavas, los objetivos principales del ejercicio fueron alcanzados con éxito.  

En términos generales, este ejercicio evidenció una correcta integración entre hardware y software, así como una adecuada gestión de temporización y respuesta a eventos. Como líneas de mejora, se puede proponer incorporar un sistema de control de volumen o modulación, implementar el comando de detención, y explorar el uso de interrupciones para optimizar la detección de teclas y la eficiencia del sistema.

Este módulo representa un ejemplo claro de la integración entre teoría de control, electrónica digital y fundamentos acústicos aplicados al desarrollo de sistemas embebidos musicales.

\subsection{Cerradura electrónica}

El desarrollo de la cerradura electrónica permitió aplicar de forma integrada los conocimientos sobre manejo de periféricos digitales, comunicación I\textsuperscript{2}C, almacenamiento no volátil y control secuencial mediante software embebido. 

El sistema demostró un funcionamiento estable, con correcta interacción entre el teclado matricial, la pantalla LCD, los indicadores luminosos y el buzzer piezoeléctrico, validando la lógica de estados implementada.

El uso de la memoria EEPROM del ATmega328P permitió conservar el PIN de acceso incluso ante la pérdida de alimentación, lo que refleja un diseño robusto y funcional para aplicaciones de control de acceso o autenticación básica. 

Asimismo, la implementación modular del código facilitó la comprensión, depuración y posterior ampliación del sistema, ya que cada bloque (LCD, teclado, EEPROM y retroalimentación) se gestionó de manera independiente.

Desde el punto de vista funcional, el sistema cumplió con todos los objetivos planteados: limitó correctamente los intentos de acceso, gestionó el bloqueo mediante una alarma visual y sonora, permitió la actualización segura del PIN y ofreció una retroalimentación clara al usuario en cada etapa del proceso. 

Como líneas de mejora, se propone incorporar un sistema de control de volumen o modulación del buzzer, implementar la gestión de temporización mediante interrupciones para optimizar la eficiencia energética, y considerar la integración de un sensor adicional (por ejemplo, magnético o de proximidad) para detectar la apertura física del mecanismo. Otra posible extensión sería implementar comunicación serial con un sistema externo para el registro de eventos o intentos fallidos de acceso.

En síntesis, la cerradura electrónica desarrollada evidencia la capacidad del microcontrolador ATmega328P para integrar entrada, procesamiento y salida de datos en tiempo real, constituyendo una aplicación representativa de sistemas embebidos orientados a la seguridad.
