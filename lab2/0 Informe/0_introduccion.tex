\section{Introducción}

El presente laboratorio tuvo como propósito afianzar los conocimientos adquiridos en el manejo de periféricos del microcontrolador ATmega328P, 
aplicando conceptos de comunicación digital, control de señales PWM, almacenamiento en memoria no volátil y desarrollo modular en lenguaje C. 
A través de distintos montajes experimentales, se buscó integrar la interacción entre hardware y software mediante el diseño de sistemas embebidos funcionales, 
optimizando tanto la organización del código como la eficiencia en el uso de recursos del microcontrolador.

Las actividades se dividieron en cuatro proyectos independientes pero complementarios entre sí: 
un \textbf{piano electrónico} con control por UART y reproducción de melodías programadas; 
una \textbf{cerradura electrónica} basada en teclado matricial, pantalla LCD y almacenamiento de PIN en EEPROM; 
un \textbf{plotter XY} controlado por motores paso a paso; 
y un \textbf{selector de color} implementado mediante un sensor LDR y un servomotor para posicionamiento angular.  
Cada uno de estos sistemas permitió poner en práctica diferentes aspectos del control digital, 
como el manejo de temporizadores, interrupciones, protocolos de comunicación y procesamiento secuencial de señales digitales.

\section{Objetivos}

\subsection{Objetivo general}

Diseñar, programar e implementar diversos sistemas embebidos utilizando el microcontrolador ATmega328P, 
integrando periféricos de entrada y salida, almacenamiento y comunicación serial, 
para afianzar los conocimientos de programación estructurada en lenguaje C y la comprensión del funcionamiento interno del hardware.

\subsection{Objetivos específicos}

\begin{itemize}
    \item Implementar un \textbf{piano electrónico} capaz de reproducir notas musicales y melodías predefinidas, 
    utilizando modulación por ancho de pulso (PWM) y comunicación UART para la selección remota de canciones.
    
    \item Desarrollar una \textbf{cerradura electrónica} con control de acceso mediante teclado matricial, 
    visualización en LCD e implementación de almacenamiento no volátil del PIN mediante la memoria EEPROM interna del microcontrolador.
    
    \item Construir un \textbf{plotter XY} controlado por motores paso a paso, 
    capaz de representar gráficamente figuras en un plano, 
    aplicando técnicas de control secuencial y temporización precisa.
    
    \item Diseñar un \textbf{selector de color} basado en un sensor fotoresistivo (LDR) 
    y un servomotor, de modo que el sistema oriente el eje de detección hacia la fuente de luz predominante.
    
    \item Analizar el comportamiento eléctrico de cada montaje, 
    evaluando la respuesta temporal, el consumo energético y la estabilidad del sistema durante su operación continua.
    
    \item Documentar el desarrollo, la metodología de diseño y los resultados experimentales de cada sistema, 
    siguiendo el formato técnico IEEE propuesto para los informes de laboratorio.
\end{itemize}

