\section{Marco Teórico}

\subsection{Piano}

El piano electrónico se basa en el microcontrolador ATmega328P, encargado de leer las entradas digitales provenientes de pulsadores y generar las notas musicales correspondientes a través de un buzzer piezoeléctrico. 
Para la síntesis de sonido, se hace uso de señales de modulación por ancho de pulso (PWM), configuradas mediante los temporizadores internos del microcontrolador, permitiendo así obtener frecuencias precisas asociadas a cada nota musical.

En términos acústicos, cada nota musical se corresponde con una frecuencia específica medida en hertzios (Hz), donde un incremento de una octava implica duplicar la frecuencia base. 
Por ejemplo, la nota La de la cuarta octava tiene una frecuencia de 440~Hz, mientras que en la quinta octava alcanza los 880~Hz. 
Esta relación logarítmica entre nota y frecuencia permite reproducir escalas musicales de manera controlada mediante la variación del período de la señal PWM.

El sistema implementa un conjunto de 12 pulsadores, cada uno asignado a una nota de la escala cromática (Do, Do\#, Re, Re\#, Mi, Fa, Fa\#, Sol, Sol\#, La, La\#, Si). 
Además, se integran dos pulsadores adicionales para modificar la octava activa, lo que amplía la capacidad tonal del instrumento sin aumentar significativamente el número de entradas físicas.

El buzzer piezoeléctrico utilizado actúa como transductor electroacústico, recibiendo la señal PWM generada por el ATmega328P y transformándola en vibraciones audibles. 
El uso de resistencias pull-up internas en los pines de entrada digital simplifica el cableado, evitando la necesidad de resistencias externas para los pulsadores.

Por otra parte, la inclusión de la comunicación serial mediante UART (Universal Asynchronous Receiver-Transmitter) permite la selección de canciones predefinidas almacenadas en memoria, 
además de ofrecer la posibilidad de enviar mensajes informativos hacia un terminal externo. 
De esta forma, el sistema no solo funciona como piano manual, sino también como reproductor automático de melodías programadas.

En cuanto al rango tonal, se implementaron las octavas 4 a 7, dado que en la práctica el buzzer presentaba limitaciones físicas en su rango de respuesta. 
Al reproducir frecuencias inferiores a la cuarta octava, el sonido resultaba monótono e indistinguible, mientras que por encima de la séptima octava el transductor no lograba generar vibración audible. 
De esta manera, el sistema cubre el rango más perceptible y estable para este tipo de actuadores.

En la sección de anexos, dentro de tablas complementarias~\ref{tab:notas_piano}, 
se detalla la correspondencia entre las notas musicales y sus frecuencias asociadas, 
así como el índice utilizado en el arreglo de notas dentro del código fuente del proyecto.

En resumen, el piano electrónico implementado combina técnicas de generación de señales mediante modulación por ancho de pulso (PWM) con el control secuencial de entradas digitales, demostrando la capacidad del microcontrolador ATmega328P para integrar procesos de adquisición, procesamiento y salida de información en tiempo real, propios de un sistema embebido interactivo.
\subsection{Cerradura electrónica}

