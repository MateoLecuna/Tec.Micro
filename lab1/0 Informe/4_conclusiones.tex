\newpage

\section{Conclusiones}

\subsection{Matriz}
La matriz de \textit{LEDs} presentó un desempeño sólido: permitió dibujar y desplazar patrones definidos en binario (unos y ceros) con buena estabilidad visual. La latencia de respuesta vía \texttt{USART} fue adecuada para la aplicación. 

\textbf{Mejoras propuestas:} Incorporar modulación por ancho de pulso (PWM) por fila o por \textit{pixel} para controlar el brillo de forma granular y habilitar imágenes/animaciones más complejas.

\subsection{Plotter}
El plotter funcionó de manera satisfactoria. El intérprete permitió programarlo de forma intuitiva y se demostró que, mediante Python, es posible generar trayectorias avanzadas con curvas suaves. Se encontró la limitación de resolución de 1 mS como movimiento mínimo de los motores del plotter, al momento de buscar mejores trazados para el circulo.

\textbf{Mejoras propuestas:} Añadir PWM para un control más fino de la velocidad de los motores; permitir la selección del modo de temporización para ajustar la resolución según la figura; evaluar perfiles de aceleración para mejorar la calidad de las trayectorias.

\subsection{Conversor}
El conversor mostró un desempeño correcto. La forma de onda obtenida fue clara y consistente con lo esperado según las gráficas del anexo~\ref{anexo:Senal_para_conversor_digital_analogo}. 

\subsection{Punzonadora}
La punzonadora operó con buen rendimiento. Se logró ajustar la carga y arrancar el proceso tanto por \texttt{USART} como mediante pulsadores físicos. El estado de la máquina y la carga seleccionada se reportaron en tiempo real por \texttt{USART}, con buena velocidad de respuesta.

\textbf{Mejoras propuestas:} Integrar sensores en la lógica de estados para que el sistema decida por eventos reales (objeto en cinta, fin de carrera) y no sólo por temporizadores; considerar enclavamientos de seguridad y manejo de fallas para aumentar la robustez.

\newpage