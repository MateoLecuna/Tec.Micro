\section{Marco Teórico}

\subsection{Microcontrolador ATmega328P}
\begin{itemize}
    \item Características principales (arquitectura AVR, memoria, periféricos).
    \item Uso de puertos GPIO para control de actuadores (motores, LEDs, relés).
    \item Temporizadores y su aplicación en control de tiempos.
    \item Comunicación serial USART (principios de transmisión y recepción de datos).
\end{itemize}


\subsection{Entradas y Salidas Digitales}
\begin{itemize}
    \item Concepto de GPIO.
    \item Uso de pulsadores como entradas digitales (debouncing si es necesario).
    \item Uso de LEDs como indicadores de estado.
\end{itemize}


\subsection{Automatización y Máquinas de Estado}
\begin{itemize}
    \item Qué es una máquina de estados finitos.
    \item Cómo se representan los estados y transiciones en un proceso automatizado (ejemplo: espera → alimentación → posicionado → punzonado → descarga → fin de ciclo).
\end{itemize}


\subsection{Control de Procesos con Cinta Transportadora y Punzonadora}
\begin{itemize}
    \item Principios básicos de una cinta transportadora en automatización.
    \item Funcionamiento de un actuador lineal/solenoide como punzón.
    \item Diferentes modos de operación según carga (ligera, media, pesada).
\end{itemize}


\subsection{Comunicación Serial (USART/UART)}
\begin{itemize}
    \item Definición y funcionamiento de UART.
    \item Ejemplos de comandos y monitoreo remoto.
    \item Aplicaciones en sistemas embebidos para interacción con el usuario o con PC.
\end{itemize}


\subsection{Conversión Digital-Analógica (DAC R-2R)}
\begin{itemize}
    \item Concepto de DAC y su importancia.\vspace{0.5em}
    
    \hspace{2em}Un DAC (Digital to Analog Converter) es un dispositivo o técnica que permite transformar valores digitales (códigos binarios) en señales analógicas (tensiones o corrientes continuas y variables en el tiempo). Su importancia radica en que la mayoría de los sistemas electrónicos trabajan de manera digital, pero el mundo físico es analógico: audio, imágenes, señales de control de motores, etc.
    
    \hspace{2em}En pocas palabras, el DAC es el “puente” entre lo digital y lo analógico.\vspace{1em}

    \item Explicación del arreglo de resistencias R-2R.\vspace{0.5em}

    \hspace{2em}El arreglo R-2R es una red de resistencias que se usa mucho para construir DACs simples y económicos. Se compone unicamente de dos valores de resistencias: una de valor R y otra de valor 2R, que se repiten en forma de escalera.
    
    \hspace{2em}Cada bit de nel número digital controla un interruptor (o transistor) que conecta la red a una referencia de tensión (Vref) o a tierra. Gracias a la proporción entre Ry 2R, la red genera tenciones que corresponden al valor binario aplicado.
    
    \hspace{2em}La ventaja del arreglo R-2R es que es fácil de implementar, no requiere de resistencias con muchos valores distintos, y mantienen una buena presición.\vspace{1em}
    
    \item Uso de una Look-Up Table (LUT) para generar señales analógicas periódicas.\vspace{0.5em}
    
    \hspace{2em}Una Look-up Table (LUT) es básicamente una tabla de valores precargada en la memoria que representa una señal digitalizada (por ejemplo, una onda seno). En lugar de calcular cada valor de la función en tiempo real, el monitor solo "lee" la tabla en orden y envía los valores a un puerto o a un DAC.
    
    \hspace{2em}Cuando esos valores se aplican de manera periódica y con la velocidad adecuada, en la salida se reconstruye una señal analógica periodica.
    
    El uso de una LUT simplifica mucho la generación de señales, por que evita cálculos complejos y garantiza que la forma de onda siempre tenga la misma calidad.\vspace{1em}
\end{itemize}


\subsection{Matrices de LEDs}
\begin{itemize}
    \item Principio de funcionamiento de una matriz de LEDs.
    \item Multiplexado y desplazamiento de mensajes.
    \item Ejemplo de uso en displays.
\end{itemize}


\subsection{Plotter y Control de Movimiento}
\begin{itemize}
    \item Concepto de plotter y su uso en ingeniería.
    \item Control de motores paso a paso o conmutados mediante relés/MOSFETs.
    \item Señales de control enviadas desde el microcontrolador a un PLC.
\end{itemize}
