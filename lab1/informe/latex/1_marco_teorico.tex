\section{Marco Teórico}

\subsection{Microcontrolador ATmega328P}
\begin{itemize}
    \item Características principales (arquitectura AVR, memoria, periféricos).
    \item Uso de puertos GPIO para control de actuadores (motores, LEDs, relés).
    \item Temporizadores y su aplicación en control de tiempos.
    \item Comunicación serial USART (principios de transmisión y recepción de datos).
\end{itemize}


\subsection{Entradas y Salidas Digitales}
\begin{itemize}
    \item Concepto de GPIO.
    \item Uso de pulsadores como entradas digitales (debouncing si es necesario).
    \item Uso de LEDs como indicadores de estado.
\end{itemize}


\subsection{Automatización y Máquinas de Estado}
\begin{itemize}
    \item Qué es una máquina de estados finitos.
    \item Cómo se representan los estados y transiciones en un proceso automatizado (ejemplo: espera → alimentación → posicionado → punzonado → descarga → fin de ciclo).
\end{itemize}


\subsection{Control de Procesos con Cinta Transportadora y Punzonadora}
\begin{itemize}
    \item Principios básicos de una cinta transportadora en automatización.
    \item Funcionamiento de un actuador lineal/solenoide como punzón.
    \item Diferentes modos de operación según carga (ligera, media, pesada).
\end{itemize}


\subsection{Comunicación Serial (USART/UART)}
\begin{itemize}
    \item Definición y funcionamiento de UART.
    \item Ejemplos de comandos y monitoreo remoto.
    \item Aplicaciones en sistemas embebidos para interacción con el usuario o con PC.
\end{itemize}


\subsection{Conversión Digital-Analógica (DAC R-2R)}
\begin{itemize}
    \item Concepto de DAC y su importancia.
    \item Explicación del arreglo de resistencias R-2R.
    \item Uso de una Look-Up Table (LUT) para generar señales analógicas periódicas.
\end{itemize}


\subsection{Matrices de LEDs}
\begin{itemize}
    \item Principio de funcionamiento de una matriz de LEDs.
    \item Multiplexado y desplazamiento de mensajes.
    \item Ejemplo de uso en displays.
\end{itemize}


\subsection{Plotter y Control de Movimiento}
\begin{itemize}
    \item Concepto de plotter y su uso en ingeniería.
    \item Control de motores paso a paso o conmutados mediante relés/MOSFETs.
    \item Señales de control enviadas desde el microcontrolador a un PLC.
\end{itemize}
