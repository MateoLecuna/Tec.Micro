\section{Conclusiones}

El desarrollo de los ejercicios propuestos en este laboratorio permitió integrar y aplicar múltiples conceptos de electrónica digital, control y comunicación de periféricos en sistemas embebidos basados en el microcontrolador \textbf{ATmega328P}.  
Cada módulo abordó un enfoque distinto del control y la interacción hardware–software, consolidando la comprensión de los principios fundamentales de adquisición, procesamiento y respuesta dentro de un entorno de tiempo real.

En el \textbf{Ejercicio A}, ...

En el \textbf{Ejercicio B}, ...

En el \textbf{Ejercicio C}, se logró implementar un control proporcional simple de velocidad y dirección de un motor DC, utilizando señales analógicas de referencia y realimentación.  
El sistema presentó un comportamiento estable y predecible, con una respuesta lineal ajustada mediante modulación PWM. Este ejercicio permitió afianzar la comprensión del ADC, del Timer1 en modo Fast PWM y del control proporcional como base para futuros controladores más complejos.

En el \textbf{Ejercicio D}, se implementó un control interactivo de una matriz de LEDs RGB mediante un joystick analógico, combinando lectura analógica, control digital temporizado y generación de color.  
La experiencia permitió comprender el funcionamiento de la comunicación unidireccional a alta velocidad (protocolo WS2812B) y la importancia de gestionar correctamente los tiempos de actualización, zonas muertas y saturación de color.  
Además, se exploró un enfoque alternativo de mapeo matricial que, si bien no fue implementado exitosamente, aportó valiosa experiencia en la gestión de coordenadas discretas y calibración de entradas analógicas.

En el \textbf{Ejercicio E}, se desarrolló un sistema de cerradura electrónica mediante identificación RFID, integrando comunicación SPI, I2C, UART y almacenamiento no volátil en EEPROM.  
El sistema permitió registrar y verificar tarjetas, mostrando mensajes en un LCD y señalizando los estados mediante LEDs.  
A pesar de que los comandos seriales no se ejecutaron completamente, el prototipo demostró un funcionamiento estable y una correcta lectura de los UID, validando la integración de múltiples periféricos en una misma arquitectura embebida.

\vspace{0.8cm}

En conjunto, las prácticas realizadas fortalecieron las habilidades de diseño, implementación y depuración de sistemas digitales, promoviendo la comprensión del trabajo modular y el uso coordinado de distintos buses de comunicación.  
Los resultados obtenidos evidencian el cumplimiento de los objetivos planteados y sientan las bases para el desarrollo de proyectos más avanzados, como sistemas de control cerrados, automatizaciones o aplicaciones de IoT basadas en microcontroladores AVR.
