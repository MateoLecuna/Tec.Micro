\documentclass[conference]{IEEEtran}
\IEEEoverridecommandlockouts\usepackage{cite}
\usepackage{amsmath,amssymb,amsfonts}
\usepackage{algorithmic}
\usepackage{graphicx}
\usepackage{textcomp}
\usepackage{xcolor}
\usepackage{array}
\usepackage{enumitem}
\usepackage{siunitx}
\usepackage[spanish]{babel}
\usepackage{multirow}
\usepackage{float}
\usepackage{booktabs}
\usepackage[hidelinks]{hyperref}
\usepackage{hhline}
\usepackage[left=2cm,right=2cm,top=2cm,bottom=2cm]{geometry}
\usepackage{listings}


\def\BibTeX{{\rm B\kern-.05em{\sc i\kern-.025em b}\kern-.08em
    T\kern-.1667em\lower.7ex\hbox{E}\kern-.125emX}}
    
\begin{document}

\title{Microcontroladores: Laboratorio 1\\
{\footnotesize \textsuperscript{}
}
\thanks{}
}

\author{\IEEEauthorblockN{1\textsuperscript{st} Mateo Lecuna}
\IEEEauthorblockA{\textit{Ingeniería en Mecatrónica} \\
\textit{Universidad Tecnológica (UTEC)}\\
Fray Bentos, Uruguay \\
mateo.lecuna@estudiantes.utec.edu.uy}
\and
\IEEEauthorblockN{2\textsuperscript{nd} Mateo Sanchez }
\IEEEauthorblockA{\textit{Ingeniería en Mecatrónica} \\
\textit{Universidad Tecnológica (UTEC)}\\
Maldonado, Uruguay \\
mateo.sanchez@estudiantes.utec.edu.uy}
}
\maketitle

\begin{abstract}
Se presenta el desarrollo de cinco sistemas embebidos implementados sobre el microcontrolador \textbf{ATmega328P}, correspondientes al Laboratorio 3 de la unidad curricular \textit{Tecnologías de Microprocesamiento}. Los ejercicios abordaron distintos escenarios de control digital y adquisición de señales, incluyendo un plóter cartesiano de dos ejes, un sistema de control de temperatura, un control de motor por referencia analógica, una matriz RGB controlada mediante joystick, y una cerradura electrónica basada en lector \textit{RFID RC522}. 

Cada proyecto integró periféricos de entrada/salida, sensores analógicos, comunicación serial UART e interfaces de visualización (LCD I2C y Python/MATLAB para gráficas). Se implementaron estrategias de control mediante modulación por ancho de pulso (PWM) y lógica secuencial programada en lenguaje C. Los resultados obtenidos demostraron el correcto funcionamiento de los sistemas tanto en simulación como en hardware físico, validando el aprendizaje y aplicación práctica de los conceptos de microprocesamiento y control embebido.

Este trabajo permitió integrar técnicas de control, comunicación y procesamiento digital en un entorno embebido real, consolidando competencias prácticas esenciales para el diseño mecatrónico.

\end{abstract}

\begin{IEEEkeywords}
ATmega328P, Sistemas Embebidos, Modulación PWM, Control de Motor, Joystick, Matriz RGB, RFID, UART, Microcontroladores.
\end{IEEEkeywords}

\section{Introducción}

\subsection{Plotter}

\subsection{Sistema de Control de Temperatura}

\subsection{Control de Motor }

\subsection{Matriz RGB con Joystick}

\subsection{Cerradura RFID}

\newpage

\section{Marco Teórico}

\subsection{Componentes y fundamentos del sistema}

\subsubsection{Microcontrolador ATmega328P / Arduino UNO}

El ATmega328P es un microcontrolador de 8 bits con arquitectura AVR, reloj típico de 16~MHz, memoria \textit{Flash} (programa), SRAM (datos) y EEPROM (no volátil). Dispone de temporizadores de 8 y 16~bits, puertos GPIO digitales, conversores A/D y periféricos de comunicación (UART, TWI/I\textsuperscript{2}C, SPI). En este laboratorio se utilizó montado sobre la plataforma Arduino~UNO, aprovechando su circuito de alimentación, cristal de 16~MHz y acceso conveniente a pines de E/S \cite{atmega328p_datasheet}.

\subsubsection{Señales PWM y temporización}

La modulación por ancho de pulso (PWM) se empleó para dos fines: (i) generación de señales acústicas en el piano (variación de frecuencia/periodo mediante temporizadores), y (ii) control de posición de servomotores (variación de ciclo útil dentro de un periodo fijo). La obtención de frecuencias precisas se realizó configurando los temporizadores del ATmega328P en modos compatibles con la resolución requerida \cite{atmega328p_datasheet}.

\subsubsection{Indicadores: LEDs y buzzer piezoeléctrico}

Se utilizaron LEDs como señalización visual de estados (éxito, error, bloqueo) y un buzzer piezoeléctrico pasivo para retroalimentación acústica. El buzzer pasivo requiere una señal conmutada (proporcionada por el microcontrolador), a diferencia del activo que integra su propio oscilador. En el piano se aprovechó PWM para tonos audibles; en la cerradura se usaron patrones de \textit{beeps} cortos/largos para codificar eventos \cite{buzzer_emx7t05sp_datasheet}.

\subsubsection{Pantalla LCD 16×2 con interfaz I\textsuperscript{2}C (PCF8574)}

Para visualización se integró un LCD 16×2 comandado en modo 4 bits mediante un \textit{backpack} basado en el expansor de E/S PCF8574. El microcontrolador opera como maestro TWI/I\textsuperscript{2}C y escribe comandos/datos en el expansor, reduciendo el número de pines necesarios y simplificando el cableado \cite{lcd_16x2_i2c_datasheet}.

\subsubsection{Teclado matricial 4×4}

El ingreso de datos en la cerradura se realizó con un teclado 4×4 escaneado por filas/columnas: se activa cada fila y se leen las columnas con \textit{pull-up} interno. Se aplicó \textit{debouncing} por software con retardos cortos y verificación del estado estable, mapeando (fila,columna) a caracteres (\texttt{0–9}, \texttt{A–D}, \texttt{*}, \texttt{\#}) \cite{keypad_4x4_okystar}.

\subsubsection{Memoria EEPROM interna}

El PIN de la cerradura se almacenó en EEPROM para persistencia sin energía. Se validó longitud y contenido al arranque, inicializando un valor por defecto cuando fuera necesario. Las rutinas de escritura/actualización minimizan el desgaste por ciclos de borrado/escritura \cite{atmega328p_datasheet}.

\subsubsection{Servomotor (modelo SG90)}

Se empleó un servomotor de rotación limitada (aprox. \(0^{\circ}\)–\(180^{\circ}\)) controlado por impulsos de \(1\text{–}2\,\mathrm{ms}\) dentro de un período de \(20\,\mathrm{ms}\) (\(50\,\mathrm{Hz}\)). El ángulo se define por el ancho del pulso: valores cercanos a \(1\,\mathrm{ms}\), \(1.5\,\mathrm{ms}\) y \(2\,\mathrm{ms}\) corresponden típicamente a los extremos e intermedio del recorrido, respectivamente. En el laboratorio se utilizó el modelo SG90, cuyas especificaciones eléctricas y mecánicas se tomaron de su hoja de datos \cite{servo_sg90_datasheet}.

\subsubsection{Motores paso a paso (Plotter)}

Para el \textit{plotter XY} se consideró el control de motores paso a paso, caracterizados por su avance angular discreto. El movimiento se obtiene excitando las bobinas según secuencias (\textit{full-/half-step}, microstepping), coordinadas temporalmente por el microcontrolador. La temporización precisa es clave para mantener torque y evitar pérdida de pasos.

\subsubsection{Sensor LDR (fotoresistencia)}

El LDR presenta una resistencia inversamente proporcional a la iluminación incidente, lo que permite sensar luminosidad con un divisor resistivo y lectura A/D. En el selector de color se empleó como señal de realimentación para posicionar el servomotor hacia la fuente de luz predominante \cite{ldr_datasheet}.

\subsubsection{Comunicación UART y \textit{debouncing}}

La UART se utilizó para emitir mensajes de diagnóstico y recibir comandos (por ejemplo, selección de melodías en el piano). Para entradas digitales provenientes de pulsadores y teclado se aplicó \textit{debouncing} por software mediante retardos y verificación consecutiva del estado estable, técnica suficiente dada la baja tasa de eventos del sistema \cite{atmega328p_datasheet}.


\subsection{Piano}

El piano se basa en el microcontrolador ATmega328P, encargado de leer las entradas digitales provenientes de pulsadores y generar las notas musicales correspondientes a través de un buzzer piezoeléctrico. 
Para la síntesis de sonido, se hace uso de señales de modulación por ancho de pulso (PWM), configuradas mediante los temporizadores internos del microcontrolador, permitiendo así obtener frecuencias precisas asociadas a cada nota musical.

En términos acústicos, cada nota musical se corresponde con una frecuencia específica medida en hertzios (Hz), donde un incremento de una octava implica duplicar la frecuencia base. 
Por ejemplo, la nota La de la cuarta octava tiene una frecuencia de 440~Hz, mientras que en la quinta octava alcanza los 880~Hz. 
Esta relación logarítmica entre nota y frecuencia permite reproducir escalas musicales de manera controlada mediante la variación del período de la señal PWM.

El sistema implementa un conjunto de 12 pulsadores, cada uno asignado a una nota de la escala cromática (Do, Do\#, Re, Re\#, Mi, Fa, Fa\#, Sol, Sol\#, La, La\#, Si). 
Además, se integran dos pulsadores adicionales para modificar la octava activa, lo que amplía la capacidad tonal del instrumento sin aumentar significativamente el número de entradas físicas.

El buzzer piezoeléctrico utilizado actúa como transductor electroacústico, recibiendo la señal PWM generada por el ATmega328P y transformándola en vibraciones audibles. 
El uso de resistencias pull-up internas en los pines de entrada digital simplifica el cableado, evitando la necesidad de resistencias externas para los pulsadores.

Por otra parte, la inclusión de la comunicación serial mediante UART (Universal Asynchronous Receiver-Transmitter) permite la selección de canciones predefinidas almacenadas en memoria, 
además de ofrecer la posibilidad de enviar mensajes informativos hacia un terminal externo. 
De esta forma, el sistema no solo funciona como piano manual, sino también como reproductor automático de melodías programadas.

En cuanto al rango tonal, se implementaron las octavas 4 a 7, dado que en la práctica el buzzer presentaba limitaciones físicas en su rango de respuesta. 
Al reproducir frecuencias inferiores a la cuarta octava, el sonido resultaba monótono e indistinguible, mientras que por encima de la séptima octava el transductor no lograba generar vibración audible. 
De esta manera, el sistema cubre el rango más perceptible y estable para este tipo de actuadores.

En la sección de anexos, dentro de tablas complementarias~\ref{tab:notas_piano}, 
se detalla la correspondencia entre las notas musicales y sus frecuencias asociadas, 
así como el índice utilizado en el arreglo de notas dentro del código fuente del proyecto.

En resumen, el piano implementado combina técnicas de generación de señales mediante modulación por ancho de pulso (PWM) con el control secuencial de entradas digitales, demostrando la capacidad del microcontrolador ATmega328P para integrar procesos de adquisición, procesamiento y salida de información en tiempo real, propios de un sistema embebido interactivo.

\subsection{Cerradura electrónica}

La cerradura digital implementada se basa en el microcontrolador ATmega328P y combina adquisición de entradas, procesamiento secuencial y señalización al usuario para validar un PIN de acceso. El sistema integra un teclado matricial 4×4 como interfaz de ingreso, una pantalla LCD 16×2 con interfaz I\textsuperscript{2}C para retroalimentación, y elementos de salida (LEDs y buzzer) para confirmar eventos y estados del sistema. El PIN se almacena de manera no volátil en la memoria EEPROM interna del microcontrolador, lo que permite conservar la configuración aun sin alimentación \cite{atmega328p_datasheet}.

\subsubsection{Ingreso por teclado matricial 4×4}
El teclado se organiza en una matriz de cuatro filas y cuatro columnas. La lectura se realiza mediante \textit{escaneo por filas}: el firmware activa secuencialmente cada fila (lógica baja) y, en cada paso, muestrea las columnas configuradas como entradas con \textit{pull-up} interno. Cuando una tecla se presiona, se establece un camino de conducción que fuerza a nivel bajo la columna correspondiente, permitiendo identificar la pareja (fila, columna) y mapearla al carácter asociado (\texttt{0 a 9}, \texttt{A a D}, \texttt{*}, \texttt{\#}). Este procedimiento se complementa con retardos breves para atenuar el efecto de rebote mecánico (\textit{debouncing}). La disposición eléctrica y el mapeo de pines siguen el módulo de teclado 4×4 utilizado \cite{keypad_4x4_okystar}.

\subsubsection{Visualización: LCD 16×2 con interfaz I\textsuperscript{2}C}
La pantalla LCD se controla en modo de 4 bits a través de un \textit{backpack} basado en el expansor de E/S PCF8574, que traslada la comunicación I\textsuperscript{2}C (líneas SDA/SCL) a señales paralelas (\texttt{RS}, \texttt{E}, \texttt{D4 a D7} y control de backlight). A nivel de bus, el ATmega328P opera como maestro TWI e inicia/termina transacciones mediante condiciones \textit{START}/\textit{STOP}, escribiendo bytes en la dirección del expansor. Sobre esa base, el software implementa las primitivas de inicialización del LCD (función, \textit{display on/off}, \textit{clear}, \textit{entry mode}) y las rutinas de posicionamiento y envío de datos/caracteres. La interfaz seleccionada reduce el número de pines GPIO necesarios y simplifica el cableado \cite{lcd_16x2_i2c_datasheet}.

\subsubsection{Almacenamiento no volátil del PIN (EEPROM)}
El PIN de usuario (longitud configurable entre 4 y 6 dígitos) se guarda en la EEPROM interna, junto con su longitud. En el arranque, si no hay un PIN válido, el firmware inicializa un valor por defecto y normaliza la estructura de almacenamiento. La EEPROM permite escritura/lectura byte a byte, con funciones de actualización que evitan ciclos de borrado innecesarios, prolongando la vida útil de la memoria \cite{atmega328p_datasheet}.

\subsubsection{Gestión de estados y seguridad}
El flujo lógico se organiza como una máquina de estados finitos:
\begin{itemize}
    \item \textbf{Modo normal}: ingreso del PIN y validación.
    \item \textbf{Cambio de PIN}: secuencia guiada en tres pasos: verificación del PIN actual, ingreso del nuevo PIN y confirmación (comando \texttt{C} para iniciar; \texttt{A} para confirmar; \texttt{B} para limpiar).
    \item \textbf{Bloqueo}: tras tres intentos fallidos consecutivos, el sistema ingresa en estado de alarma/bloqueo. La salida requiere la \textit{llave maestra} (\texttt{D}), lo que restablece el contador de intentos y retorna a modo normal.
\end{itemize}
Adicionalmente, se implementan validaciones de formato (por ejemplo, longitud mínima de 4 dígitos). Las decisiones de diseño (límite de intentos, clave maestra, confirmación doble del nuevo PIN) elevan la robustez frente a errores de ingreso y mitigaciones de prueba y error.

\subsubsection{Señalización: LEDs y Buzzer}
La interacción con el usuario se refuerza mediante señales visuales y acústicas. El \textbf{LED verde} y un patrón de tres pulsos cortos del \textbf{buzzer} indican validación correcta; el \textbf{LED rojo} junto con un pulso sostenido del buzzer señalan error de PIN o formatos inválidos. En estado de bloqueo, se alternan periodos de activación/desactivación de LED rojo y buzzer hasta la intervención con la tecla maestra. El buzzer piezoeléctrico empleado funciona como transductor binario (encendido/apagado) para retroalimentación breve, diferenciándose del uso tonal continuo aplicado en el ejercicio del piano \cite{buzzer_emx7t05sp_datasheet}.

\subsubsection{Consideraciones temporales y \textit{debouncing}}
La temporización se apoya en retardos de corta duración para estabilizar lecturas del teclado y construir patrones de señalización perceptibles (confirmaciones, alarmas). En sistemas embebidos, el \textit{debouncing} por software con retardos y verificación consecutiva de estado es una estrategia suficiente para teclados de baja tasa de eventos; no obstante, para escalabilidad se podrían considerar interrupciones por cambio de pin, filtros digitales o máquinas de tiempo dedicadas.

\subsubsection{Síntesis}
En conjunto, la cerradura digital ilustra la integración de periféricos síncronos (I\textsuperscript{2}C/TWI) y asíncronos (GPIO escaneados), memoria no volátil (EEPROM) y una máquina de estados que implementa políticas de seguridad (longitud mínima, límite de intentos, clave maestra). El diseño resultante evidencia la capacidad del ATmega328P para articular adquisición, procesamiento y respuesta en tiempo real en un entorno embebido con interfaz hombre–máquina clara y consistente.



\section{Metodología}
\subsection{Materiales}

\newpage

\section{Resultados}
\subsection{Conversor digital-análogo}

\begin{verbatim}
RESET:
    ; Apuntar Z al inicio de la LUT
    ldi ZH, HIGH(LUT_START<<1)
    ldi ZL, LOW(LUT_START<<1)

    ; Apuntar Y al final de la LUT
    ldi YH, HIGH(LUT_END<<1)
    ldi YL, LOW(LUT_END<<1)
\end{verbatim}

\begin{verbatim}
MAIN_LOOP:
    ; Leer siguiente valor de la LUT
    ; y avanzar puntero
    lpm r16, Z+
    out PORTD, r16
    rcall delay

    cp  ZL, YL
    cpc ZH, YH
    ; Si no es el fin, seguir
    brne MAIN_LOOP

    ; Volver al inicio de la tabla
    ldi ZH, HIGH(LUT_START<<1)
    ldi ZL, LOW(LUT_START<<1)
    rjmp MAIN_LOOP
\end{verbatim}

\begin{figure}[H]
  \centering
  \includegraphics[width=\linewidth]{./Anexos/Resultados/DAC/Circuito.jpg}
  \caption{Circuito final para conversor digital-análogo. Fuente: \cite{LabDrive}.}
  \label{fig:conversor_circuito}
\end{figure}

\begin{figure}[H]
  \centering
  \includegraphics[width=\linewidth]{./Anexos/Resultados/DAC/Ocsiloscopio.jpg}
  \caption{visualización de señal en osciloscopio. Fuente: \cite{LabDrive}.}
  \label{fig:conversor_osciloscopio}
\end{figure}

\subsection{Matriz}

\subsubsection{Mapeado de puertos y pines}
Se mapean los puertos y pines individuales del mismo modo al que se puede apreciar en el anexo \ref{anexo:Look_Up_Table}. 

\subsubsection{Encendido de un solo LED}
Utilizando la LUT, se recorre añadiendo la posición especificada previamente en los registros \texttt{row} y \texttt{col}, para encender especificamente el LED mapeado a esa fila y columna. La función hace uso de \texttt{CLEAR\_LED} y \texttt{SET\_LED} (Véase anexo \ref{anexo:Bit_Masks}) para encender la fila y columna respectivamente, sin modificar el resto de bits en los puertos. 

\begin{verbatim}
ldi ZH, high(ROW_PORTS<<1) 
ldi ZL, low(ROW_PORTS<<1)
add ZL, row adc ZH, r1  
lpm r16, Z ; r16 = row port adress
        
ldi ZH, high(ROW_MASKS<<1) 
ldi ZL, low(ROW_MASKS<<1)
add ZL, row adc ZH, r1
lpm r17, Z ; r18 = row pin mask

; Encender fila
clr ZH mov ZL, r16 
mov r16, r17
rcall CLEAR_BIT

ldi ZH, high(COL_PORTS<<1) 
ldi ZL, low(COL_PORTS<<1)  
add ZL, col adc ZH, r1  
lpm r16, Z ; r16 = column port adress

ldi ZH, high(COL_MASKS<<1) 
ldi ZL, low(COL_MASKS<<1)  
add ZL, col adc ZH, r1  
lpm r17, Z ; r17 = column pin mask

; Encender columna
clr ZH mov ZL, r16 
mov r16, r17
rcall SET_BIT
\end{verbatim}

\subsubsection{Multiplexado y dibujo de cuadros}
Al encender un LED solo muy rápido se realizan operaciones condicionales con una máscara guardada en FLASH que representa la imágen del fotograma. Cada LED individual es comparado utilizando \texttt{r16}, el cual guarda la máscara posicional con la cual se realiza una operación \texttt{AND} para identificar si el LED se prendería en ese fotograma o no. 

\begin{verbatim}
ldi row, 0 RENDER_FRAME_ROW_LOOP:  
ldi r16, 0b00000001 ; Frame mask
lpm r17, Z+

ldi col, 0 RENDER_FRAME_COL_LOOP:
    rcall CLEAR_MATRIX

    push r16
    and r16, r17

    cpi r16, 0 
    breq RENDER_FRAME_SKIP_LED

    rcall TURN_LED
    rcall TEST_DELAY

    RENDER_FRAME_SKIP_LED:
    pop r16
    lsl r16
inc col cpi col, 8 
brlo RENDER_FRAME_COL_LOOP 

inc row cpi row, 8 
brlo RENDER_FRAME_ROW_LOOP
\end{verbatim}

\subsubsection{USART}
Utilizando la interrupción de overflow del \texttt{TIMER2} configurada para 100ms, se pueden realizar animaciones de texto desplazante, por ejemplo, al incrementar la posición del puntero del fotograma actual dentro de la misma interrupción. Utilizando lógica de estados Y USART asíncrono con un buffer de anillo (véase anexo \ref{anexo:USART_Asincrono_con_Ring_Buffer}), se programan las diferentes funcionalidades menú de inicio, mostrar un texto desplazante o mostrar difernetes imágenes en la matriz (véase anexo \ref{anexo:Maquina_de_Estados}).


\begin{figure}[H]
  \centering
  \includegraphics[width=0.7\linewidth]{./Anexos/Resultados/Matriz/Circuito.jpg}
  \caption{Circuito final para Matriz de LEDs. Fuente: \cite{LabDrive}.}
  \label{fig:circuito_matriz}
\end{figure}

\begin{figure}[H]
  \centering
  \includegraphics[width=0.7\linewidth]{./Anexos/Resultados/Matriz/Menu.jpg}
  \caption{Menu de opciones para Matriz de LEDs. Fuente: \cite{LabDrive}.}
  \label{fig:menu_matriz}
\end{figure}



\subsection{Punzonadora}

\subsubsection{Configuración de temporizador variable}
Se utilizan macros para realizar el control de temporizadores. En este caso se muestra como se configura al \text{TIMER1} para realizar una determinada cantidad de overflows de 1 segundo cada uno. 

\begin{verbatim}
.macro ENABLE_TIMER_1
; @0 Timer seconds
push r16
mov timer1_ovf_counter, @0
ldi r16, 0b101		 sts TCCR1B, r16
ldi r16, HIGH(49911) sts TCNT1H, r16
ldi r16, LOW(49911)	 sts TCNT1L, r16 
ldi r16, (1<<TOV1)   out TIFR1,  r16 
ldi r16, (1<<TOIE1)  sts TIMSK1, r16 
pop r16
.endmacro
\end{verbatim}


\subsubsection{Manejo de estados}
El manejo de estados del sistema es una máquina de estados como la que se puede ver en el anexo \ref{anexo:Maquina_de_Estados}. Aquí se muestra un fragmento con los estados internos:

Control de estados general:
\begin{verbatim}
STATE_MACHINE:
cpi state, 0 breq STATE_MACHINE_STOP
cpi state, 1 breq STATE_MACHINE_ADVANCE
cpi state, 2 breq STATE_MACHINE_WAIT_1
cpi state, 3 breq STATE_MACHINE_PUNCH
cpi state, 4 breq STATE_MACHINE_WAIT_2
cpi state, 5 breq STATE_MACHINE_EXTRACT
rjmp STATE_MACHINE_END
\end{verbatim}

Luego en cada estado individual existe una serie de condicionales que determinará el tiempo con el que se confgiura \texttt{TIMER1} para pasar al siguiente estado, dependiendo de la carga configurada.
\begin{verbatim}
STATE_MACHINE_STOP:
cpi load, 0 breq STOP_LOAD_0
cpi load, 1 breq STOP_LOAD_1
cpi load, 2 breq STOP_LOAD_2
rjmp STATE_MACHINE_STOP_SKIP
\end{verbatim}

\subsubsection{Botones, LEDs,  y debouncing}
En Standby se enciende el indicador izquierdo verde, indicando que el sistema está esperando la señal de arranque, ya sea por USART o por los pulsadores físicos. Con el pulsador de la izquierda se maneja el cambio de carga, el cual hace un ciclo de cambio entre liviana, mediana, y pesada, indicadas por LEDs de color verde, amarillo, y rojo, respectivamente.

En este fragmento de código se muestra como se maneja de deobuncing, y pausado de entradas externas durante el funcionamiento de la máquina: cuando esta arranca su funcionamiento se desabilitan las interrupcioens externas, y las interrupciones de recepción de USART. Se vuelven a habilitar una vez la máquina de estados reconoce que llegó al final de la secuencia. Se implementaron macros para la habilitación y deshabilitación de interrupciones.

\begin{verbatim}
INT0_ISR:
    push r16
    in r16, SREG
    push r16 

    DISABLE_BUTTONS
    DISABLE_RX

    ldi state, 1
    ldi r16, 1 
    sts event_pending, r16

    pop r16
    out SREG, r16
    pop r16 
    reti
\end{verbatim}

Temporizador de deboucing para botones de cambio de carga. Timer 2 configurado para un overflow de aproximadamente 1 segundo (usando contador de overflow externo):

\begin{verbatim}
    T2_OVF_ISR:
	push r16 
    in r16, SREG 
	push r16 
	
	inc timer2_ovf_counter
	
	ldi r16, _TIMER2_OVF_COUNT 
    cp r16, timer2_ovf_counter 
    brsh T2_OVF_ISR_END 
    
	DISABLE_TIMER_2
	ENABLE_BUTTONS
	clr timer2_ovf_counter
    
    T2_OVF_ISR_END:
		pop r16
		out SREG, r16
		pop r16	
		reti

\end{verbatim}


\subsubsection{USART}
Identico para los otros casos, se configura un menú que es enviado en el RESET del programa para ser mostrado al inicio, y una serie de condicionales determinan el comportamiento del sistema dependiendo del comando que recibe. En este caso: 1, 2, y 3 para seleccionar el tipo de carga. Y A para iniciar la secuencia del programa.


\begin{figure}[H]
  \centering
  \includegraphics[width=0.7\linewidth]{./Anexos/Resultados/Punzonadora/Circuito.jpg}
  \caption{Ensamblado final de punzonadora. Fuente: \cite{LabDrive}.}
  \label{fig:punzonadora_circuito}
\end{figure}

\subsection{Plotter}

Utilizando \texttt{.equ} se mapean las instrucciones de la tabla a un nombre fácil de interpretar. La ventaja que se tiene al tratarse de bits individuales, es que luego se podrán realizar operaciones como \texttt{UP + RIGHT} para comandar al plotter el movimeinto arriba a la derecha de manera más legible e intuitiva.

\subsubsection{Mapeo de comandos}
\begin{verbatim}
.equ SOLENOID_DOWN =  0b00000100
.equ SOLENOID_UP   =  0b00001000
.equ DOWN          =  0b00010000
.equ UP            =  0b00100000
.equ RIGHT         =  0b01000000
.equ LEFT          =  0b10000000
.equ STOP          =  0b00000000
\end{verbatim}

El siguiente ejemplo es de un programa hecho para el interprete creado para el plotter:
\begin{verbatim}
TRIANGLE_DATA:
    .db 5,  SOLENOID_DOWN	
    .db 20, SOLENOID_DOWN + RIGHT			
    .db 10, SOLENOID_DOWN + UP + LEFT		
    .db 10, SOLENOID_DOWN + DOWN + LEFT
    .db 5,  SOLENOID_UP		
    .db 5,  STOP
\end{verbatim}

\subsubsection{Interprete de secuencias programadas}

El interprete \texttt{DRAW} se encarga de escribir el valor guardado en la secuencia en \texttt{PORTD}, y esperar la cantidad de tiempo indicada en la misma antes de pasar a la siguiente instrucción.

\begin{verbatim}
DRAW:
    mov ZL, r16 mov ZH, r17
    DRAW_LOOP:
    lpm r18, Z+ ; Time
    lpm r19, Z+ ; Instruction

    out PORTD, r19 ; Send instruction

    cpi r19, STOP
    breq DRAW_END ; Stop drawing

    DRAW_TIMER_LOOP: 
    rcall S1 ; Timer
    dec r18 brne DRAW_TIMER_LOOP 

    rjmp DRAW_LOOP ; Next instruction

    DRAW_END:
    ret
\end{verbatim}
Se encontró que para temporizadores con valores menores a 1ms no permitían el movimiento adecuado de los motores, por lo cual se decidió limitar la resolución con un temporizador de 1500 $\mu S$

\subsubsection{USART}
Luego se utiliza la funcionalidad de USART asíncrono con Ring Buffer mostrada en el anexo \ref{anexo:USART_Asincrono_con_Ring_Buffer}, para mostrar un menú en el inicio del programa y darle la opción al usuario de elegir entre diferentes comandos para el plotter. Se agregaron las funcionalidades \texttt{ARRIBA, ABAJO, IZQUIERDA, DERECHA} para permitir posicionar el solenoide del plotter antes de realizar el trazado de las figuras.

\begin{figure}[H]
  \centering
  \includegraphics[width=0.7\linewidth]{./Anexos/Resultados/Plotter/Menu.jpg}
  \caption{Menu en USART para el Plotter. Fuente: \cite{LabDrive}.}
  \label{fig:plotter_menu}
\end{figure}

\begin{figure}[H]
  \centering
  \includegraphics[width=\linewidth]{./Anexos/Resultados/Plotter/Dibujos.jpg}
  \caption{Figuras dibujadas en el plotter. Fuente: \cite{LabDrive}.}
  \label{fig:plotter_figuras}
\end{figure}

\section{Conclusiones}

El desarrollo de los ejercicios propuestos en este laboratorio permitió integrar y aplicar múltiples conceptos de electrónica digital, control y comunicación de periféricos en sistemas embebidos basados en el microcontrolador \textbf{ATmega328P}.  
Cada módulo abordó un enfoque distinto del control y la interacción hardware–software, consolidando la comprensión de los principios fundamentales de adquisición, procesamiento y respuesta dentro de un entorno de tiempo real.

En el \textbf{Ejercicio A}, Habiendo llevado a cabo la consigna solicitada, cabe destacar la experiencia en programación con C AVR adquirida, además del conocimiento absorbido respecto al manejo y programación de Máquinas de Control Numérico Computarizado.\\
    Además de obtener conocimiento práctico en resolución de problemas, fué posible realizar un diagnóstico mecánico de las piezas físicas y los problemas presentados por la máquina, así como explorar posibles mejoras al sistema (como la adición de un final de carrera en el eje X.
Se lograron dibujar múltiples figuras haciendo uso de la máquina.

En el \textbf{Ejercicio B}, ...

En el \textbf{Ejercicio C}, se logró implementar un control proporcional simple de velocidad y dirección de un motor de corriente continua, utilizando señales analógicas de referencia y realimentación.  
El sistema presentó un comportamiento estable y predecible, con una respuesta lineal ajustada mediante modulación PWM.  
Este ejercicio permitió afianzar la comprensión del funcionamiento del ADC, del temporizador \texttt{Timer1} en modo \textit{Fast PWM} y del control proporcional como base para futuros controladores más complejos.

En el \textbf{Ejercicio D}, se implementó un control interactivo de una matriz de LEDs RGB mediante un joystick analógico, combinando lectura analógica, control digital temporizado y generación de color.  
La experiencia permitió comprender el funcionamiento de la comunicación unidireccional a alta velocidad (protocolo \textbf{WS2812B}) y la importancia de gestionar correctamente los tiempos de actualización, zonas muertas y saturación de color.  
Además, se exploró un enfoque alternativo de mapeo matricial que, si bien no se implementó exitosamente, aportó experiencia valiosa en la gestión de coordenadas discretas y calibración de entradas analógicas.

En el \textbf{Ejercicio E}, se desarrolló una cerradura electrónica mediante identificación \textbf{RFID}, integrando comunicación SPI, I2C, UART y almacenamiento no volátil en EEPROM.  
El sistema permitió registrar y verificar tarjetas, mostrando mensajes en un LCD y señalizando los estados mediante LEDs.  
A pesar de que los comandos seriales no se ejecutaron completamente, el prototipo demostró un funcionamiento estable y una correcta lectura de los UID, validando la integración de múltiples periféricos en una misma arquitectura embebida.

\vspace{0.8cm}

En conjunto, las prácticas realizadas fortalecieron las habilidades de diseño, implementación y depuración de sistemas digitales, promoviendo la comprensión del trabajo modular y el uso coordinado de distintos buses de comunicación.  
Los resultados obtenidos evidencian el cumplimiento de los objetivos planteados y sientan las bases para el desarrollo de proyectos más avanzados, como sistemas de control en lazo cerrado, automatizaciones o aplicaciones de \textbf{IoT} basadas en microcontroladores AVR.

            
\bibliographystyle{IEEEtran}
\nocite{*}
\bibliography{5_referencias}

\newpage

\section{Anexos}

\subsection{Esquemas de conexión}

\subsubsection{Piano electrónico}

En la Figura~\ref{fig:conexion_piano} se muestra el esquema de conexión del piano electrónico, elaborado en Tinkercad®. 
Se pueden observar las conexiones entre el microcontrolador ATmega328P, los pulsadores, y el buzzer piezoeléctrico pasivo EMX-7T05SP.

\begin{figure}[H]
    \centering
    \includegraphics[width=0.5\textwidth]{Anexos/Conexionado_de_piano.pdf}
    \caption{Esquema de conexión del piano electrónico. Elavoración propia en Tinkercad®.}
    \label{fig:conexion_piano}
\end{figure}

\subsubsection{Cerradura electrónica}

En la Figura~\ref{fig:Cerradura_electronica} se presenta el esquema de conexión del sistema de cerradura electrónica. 
El circuito fue diseñado en Tinkercad® y muestra la interconexión entre el microcontrolador ATmega328P, el teclado matricial 4×4, 
la pantalla LCD 16×2 con interfaz I²C, los LEDs indicadores (rojo y verde) y el buzzer de señalización.

\begin{figure}[H]
    \centering
    \includegraphics[width=0.5\textwidth]{Anexos/Cerradura_electrónica.pdf}
    \caption{Esquema de conexión del candado electrónico. Elaboración propia en Tinkercad®.}
    \label{fig:Cerradura_electronica}
\end{figure}

\subsection{Códigos fuente}

\subsubsection{Piano electrónico}
A continuación se detalla la estructura de archivos del proyecto del piano electrónico:

\begin{verbatim}
-- main.c
-- piano.h
-- piano.c
-- sonidos.h
-- sonidos.c
-- funciones.h
-- funciones.c
-- figuras.h
-- figuras.c
-- canciones.h
-- canciones.c
-- hw_pins.h
-- hw_pins.c
-- uart.h
-- uart.c

\end{verbatim}

El código fuente completo del proyecto se encuentra disponible en el repositorio de GitHub \cite{utec_tecmicro}.

\subsection{Evidencias}

Las evidencias de la realización de los ejercicios del laboratorio dos, se encuentran adjuntas en la carpeta \textit{“Evidencias”} \cite{github_evidencias_lab2}. Dentro de ella, se encuentran las fotos, videos, documentos, que demuestran la correcta implementación y funcionamiento de los sistemas desarrollados en este laboratorio.

\vspace{0.3cm}

\subsection{Tablas complementarias}

A continuación se presentan las tablas complementarias las cuales resumen la información técnica relevante utilizada durante el desarrollo y programación de los sistemas.

\vspace{0.5cm}

\subsubsection{Piano}

\paragraph{Correspondencia entre notas y frecuencias del piano}

En la Tabla~\ref{tab:notas_piano} se presenta la correspondencia entre las notas musicales, 
sus frecuencias y el índice utilizado dentro del arreglo \texttt{notas[]} en el programa del piano electrónico. 
Los valores corresponden a la octava número 4, considerada la octava base del sistema. 

\begin{table}[H]
\centering
\begin{tabular}{|c|c|c|c|}
\hline
\textbf{Nota} & \textbf{Frecuencia (Hz)} & \textbf{Índice en arreglo} & \textbf{Octava} \\
\hline
DO  & 262 & 0  & 4 \\
DO\# & 277 & 1  & 4 \\
RE  & 294 & 2  & 4 \\
RE\# & 311 & 3  & 4 \\
MI  & 330 & 4  & 4 \\
FA  & 349 & 5  & 4 \\
FA\# & 370 & 6  & 4 \\
SOL & 392 & 7  & 4 \\
SOL\# & 415 & 8  & 4 \\
LA  & 440 & 9  & 4 \\
LA\# & 466 & 10 & 4 \\
SI  & 494 & 11 & 4 \\
\hline
\end{tabular}
\caption{Correspondencia entre notas, frecuencias e índices de la octava 4 utilizada en el piano electrónico.}
\label{tab:notas_piano}
\end{table}

Las frecuencias de las octavas superiores (5, 6 y 7) se obtuvieron a partir de las frecuencias de referencia de la octava 4, 
utilizando la relación:

\[
f_{n} = f_{0} \times 2^{n}
\]

donde \( f_{n} \) representa la frecuencia de una nota en la octava \( n \), 
y \( f_{0} \) es la frecuencia de la misma nota en la octava base. 
De esta forma, el sistema puede reproducir diferentes rangos tonales de manera programática.

\vspace{0.5cm}

\paragraph{Mapa de pines del piano electrónico}

En la Tabla~\ref{tab:pines_piano} se detalla la asignación de pines del microcontrolador ATmega328P utilizados en el proyecto del piano electrónico, incluyendo las funciones específicas de cada pin.

\begin{table}[H]
\centering
\resizebox{\columnwidth}{!}{%
\begin{tabular}{|c|c|c|}
\hline
\textbf{Señal} & \textbf{Pin Arduino} & \textbf{Descripción} \\
\hline
BUZZER (PWM) & D3 (OC2B) & Salida PWM para tono (Timer2). \\
UART\_TX     & D1 (TX)    & Transmisión serie a 9600 baud. \\
UART\_RX     & D0 (RX)    & Recepción serie (comandos). \\
TECLA 1..12  & D2, D4..D9, A0..A3 & Entradas con resistencias pull-up internas. \\
\hline
\end{tabular}
}
\caption{Asignación de pines utilizada en el piano electrónico.}
\label{tab:pines_piano}
\end{table}

\paragraph{Comandos UART implementados}

Los comandos UART disponibles para el control del piano electrónico se presentan en la Tabla~\ref{tab:uart_piano}. 

\begin{table}[H]
\centering
\resizebox{\columnwidth}{!}{%
\begin{tabular}{|c|c|c|}
\hline
\textbf{Comando} & \textbf{Acción} & \textbf{Observación} \\
\hline
\texttt{C1}    & Reproduce canción 1. & Melodía predefinida en \texttt{canciones.c}. \\
\texttt{C2}    & Reproduce canción 2. & Melodía predefinida en \texttt{canciones.c}. \\
\texttt{PIANO} & Modo manual. & Habilita lectura de teclas 1..12. \\
\hline
\end{tabular}
}
\caption{Comandos UART disponibles para el control del piano electrónico.}
\label{tab:uart_piano}
\end{table}

Estos comandos permiten al usuario seleccionar entre la reproducción automática de melodías predefinidas o el modo manual para tocar notas individuales mediante los pulsadores conectados al microcontrolador.

\vspace{0.5cm}

\paragraph{Figuras musicales y duración}

En la Tabla~\ref{tab:figuras_piano} se muestra la relación de duraciones empleada para programar las melodías en el piano.

\begin{table}[H]
\centering
\begin{tabular}{|l|c|c|}
\hline
\textbf{Figura} & \textbf{Relación vs negra} & \textbf{Ejemplo (ms) si negra = 500 ms} \\
\hline
Redonda  & $\times 4$   & 2000 \\
Blanca   & $\times 2$   & 1000 \\
Negra    & $\times 1$   & 500 \\
Corchea  & $\times 1/2$ & 250 \\
Semicor. & $\times 1/4$ & 125 \\
Semisemi. & $\times 1/8$ & 62.5 \\
\hline
\end{tabular}
\caption{Relación de duraciones empleada para programar las melodías.}
\label{tab:figuras_piano}
\end{table}

Estas duraciones permiten definir el ritmo de las melodías reproducidas por el piano electrónico, facilitando la programación de diferentes estilos musicales.

\newpage

\subsubsection{Cerradura electrónica}

\paragraph{Mapa del teclado matricial 4×4}

En la Tabla~\ref{tab:teclado_cerradura} se detalla la disposición de las teclas del teclado matricial 4×4 utilizado en la cerradura electrónica, 
así como las funciones asociadas a cada una dentro del sistema.  
Esta asignación permitió simplificar la interacción del usuario con el microcontrolador y establecer un conjunto claro de comandos de control.

\begin{table}[H]
\centering
\resizebox{\columnwidth}{!}{%
\begin{tabular}{|c|c|c|c|c|}
\hline
\textbf{Fila/Columna} & \textbf{C0} & \textbf{C1} & \textbf{C2} & \textbf{C3} \\
\hline
\textbf{F0} & 1 & 2 & 3 & A (Confirmar) \\
\hline
\textbf{F1} & 4 & 5 & 6 & B (Borrar) \\
\hline
\textbf{F2} & 7 & 8 & 9 & C (Cambiar PIN) \\
\hline
\textbf{F3} & * & 0 & \# & D (Llave maestra) \\
\hline
\end{tabular}
}
\caption{Distribución y funciones asignadas a cada tecla del teclado matricial 4×4. Fuente: elaboración propia.}
\label{tab:teclado_cerradura}
\end{table}

El mapeo se mantuvo constante durante todas las pruebas, garantizando la coherencia entre el hardware y la lógica del programa, 
y permitiendo una detección precisa de cada pulsación sin interferencias entre filas y columnas.

\vspace{0.3cm}

\paragraph{Mensajes mostrados en pantalla LCD}

La Tabla~\ref{tab:lcd_cerradura} resume los mensajes principales que el sistema muestra en el LCD 16×2 durante las distintas etapas de operación.  
Cada texto cumple una función específica de retroalimentación para el usuario, 
indicando tanto los estados normales de funcionamiento como las condiciones de error o advertencia.

\begin{table}[H]
\centering
\resizebox{\columnwidth}{!}{%
\begin{tabular}{|l|l|}
\hline
\textbf{Estado del sistema} & \textbf{Mensaje en pantalla LCD} \\
\hline
Inicio del sistema & \texttt{Pin:} \\
PIN correcto & \texttt{OK} \\
PIN incorrecto & \texttt{Error} \\
Longitud insuficiente & \texttt{Min 4 digitos} \\
Bloqueo por intentos fallidos & \texttt{Bloqueado} \\
Modo cambio de PIN (actual) & \texttt{Ingrese actual} \\
Modo cambio de PIN (nuevo) & \texttt{Nuevo PIN (4-6)} \\
Modo confirmación de PIN & \texttt{Confirmar nuevo} \\
PIN actualizado & \texttt{PIN actualizado} \\
Reinicio por llave maestra & \texttt{MASTER OK} \\
\hline
\end{tabular}
}
\caption{Mensajes visualizados en la pantalla LCD durante la operación del sistema. Fuente: elaboración propia.}
\label{tab:lcd_cerradura}
\end{table}

Estos mensajes facilitaron la interacción usuario–sistema, 
brindando indicaciones claras sobre cada fase del proceso de validación o configuración 
y reduciendo la posibilidad de errores en la manipulación del dispositivo.

\vspace{0.3cm}

\paragraph{Comportamiento de LEDs y buzzer}

La Tabla~\ref{tab:leds_buzzer_cerradura} describe la correspondencia entre los eventos del sistema 
y las señales de salida emitidas por los indicadores luminosos (LEDs) y el buzzer piezoeléctrico.  
Estas salidas fueron fundamentales para proporcionar una retroalimentación inmediata, tanto visual como sonora, 
sobre el estado actual de la cerradura.

\begin{table}[H]
\centering
\resizebox{\columnwidth}{!}{%
\begin{tabular}{|l|c|c|l|}
\hline
\textbf{Evento} & \textbf{LED Verde} & \textbf{LED Rojo} & \textbf{Buzzer} \\
\hline
PIN correcto & Encendido breve & Apagado & 3 pulsos cortos \\
PIN incorrecto & Apagado & Encendido 0.5 s & 1 tono continuo 0.5 s \\
PIN demasiado corto & Apagado & Encendido breve & 1 pitido corto \\
Bloqueo & Apagado & Intermitente & Alarma intermitente \\
Cambio de PIN exitoso & Encendido breve & Apagado & 3 pulsos cortos \\
Cambio de PIN erróneo & Apagado & Encendido breve & 1 pitido corto \\
Reinicio (tecla D) & Encendido breve & Apagado & 3 pulsos cortos \\
\hline
\end{tabular}
}
\caption{Relación entre eventos del sistema y las salidas de retroalimentación (LEDs y buzzer). Fuente: elaboración propia.}
\label{tab:leds_buzzer_cerradura}
\end{table}

El esquema de respuesta permitió distinguir fácilmente entre acciones exitosas, errores y condiciones de alarma, 
mejorando la usabilidad del sistema y confirmando la correcta sincronización entre hardware y software.


\end{document}