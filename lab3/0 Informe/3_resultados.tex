\section{Resultados}

\subsection{Plotter}

\subsection{Sistema de Control de Temperatura}

\subsection{Control de Motor}

Durante las pruebas del sistema de control proporcional, se registraron los valores de referencia, realimentación, ciclo de trabajo del PWM y dirección de giro, transmitidos mediante UART en formato CSV.  
Los datos se visualizaron en tiempo real mediante una aplicación en \textit{Python}, lo que permitió verificar la respuesta dinámica del sistema.

Se observó una relación proporcional estable entre la diferencia de tensiones de los potenciómetros y la velocidad del motor. En la zona muerta (±10 cuentas ADC) el motor permaneció detenido, y al aumentar la diferencia de referencia la velocidad respondió suavemente en ambos sentidos.  
El valor mínimo de PWM efectivo (180) permitió superar el par de arranque bajo carga, evitando bloqueos.

\begin{figure}[H]
    \centering
    \includegraphics[width=0.85\linewidth]{Anexos/C1.jpg}
    \caption{Gráfico obtenido en tiempo real mediante Python, mostrando la relación entre la posición de los potenciómetros y el valor del PWM aplicado al motor.}
    \label{fig:control_motor}
\end{figure}

\subsection{Matriz RGB con Joystick}

El sistema respondió correctamente a los desplazamientos del joystick en ambos ejes, mostrando el movimiento del LED activo dentro de la matriz 8×8.  
La zona muerta definida permitió eliminar falsas detecciones y estabilizar la posición cuando el joystick se encontraba en reposo.  
El retardo (\textit{cooldown}) aplicado entre lecturas contribuyó a una sensación de desplazamiento controlada y sin parpadeos.

Al presionar el botón del joystick, el color del LED activo cambiaba de forma pseudoaleatoria, generando distintos tonos visibles en pantalla.  
Aunque se intentó implementar un mapeo matricial continuo, la solución final basada en condiciones discretas resultó más sencilla y confiable.

\begin{figure}[H]
    \centering
    \includegraphics[width=0.8\linewidth]{Anexos/D1.jpg}
    \caption{Desplazamiento del LED activo dentro de la matriz RGB controlado mediante el joystick.}
    \label{fig:matriz_movimiento}
\end{figure}

\begin{figure}[H]
    \centering
    \includegraphics[width=0.8\linewidth]{Anexos/D2.jpg}
    \caption{Cambio de color del LED activo al presionar el botón del joystick.}
    \label{fig:matriz_color}
\end{figure}

\subsection{Cerradura RFID}

El sistema RFID demostró un funcionamiento estable en la lectura y verificación de tarjetas MIFARE.  
Durante las pruebas, el LCD mostró correctamente los mensajes de estado (“Acerque tarjeta”, “Acceso concedido”, “Acceso denegado”), y los LEDs indicaron visualmente la validez o rechazo del acceso.  
Las tarjetas registradas permanecieron almacenadas en la memoria EEPROM incluso tras reiniciar el sistema, comprobando la persistencia de datos.  

Aunque los comandos UART de administración no se interpretaron correctamente, el sistema respondió con mensajes informativos (“Comandos: ADD, DEL, LIST”) ante cualquier entrada, lo que permitió verificar la comunicación serial básica.  
La detección en el modo de programación resultó más sensible a los tiempos de lectura, pero se estabilizó ajustando los \textit{delays} entre consultas.

\begin{figure}[H]
    \centering
    \includegraphics[width=0.8\linewidth]{Anexos/E1.jpg}
    \caption{Pantalla inicial del sistema RFID con mensaje de espera para detección de tarjeta.}
    \label{fig:rfid_inicio}
\end{figure}

\begin{figure}[H]
    \centering
    \includegraphics[width=0.8\linewidth]{Anexos/E2.jpg}
    \caption{Validación correcta: acceso concedido y encendido del LED verde.}
    \label{fig:rfid_ok}
\end{figure}

\begin{figure}[H]
    \centering
    \includegraphics[width=0.8\linewidth]{Anexos/E3.jpg}
    \caption{Lectura de tarjeta no registrada: acceso denegado y encendido del LED rojo.}
    \label{fig:rfid_denegado}
\end{figure}

\begin{figure}[H]
    \centering
    \includegraphics[width=0.8\linewidth]{Anexos/E4.jpg}
    \caption{Modo programación: interfaz para alta y baja de tarjetas.}
    \label{fig:rfid_prog}
\end{figure}

\begin{figure}[H]
    \centering
    \includegraphics[width=0.8\linewidth]{Anexos/E6.jpg}
    \caption{Pantalla correspondiente a la opción de agregar (\texttt{ADD}) una nueva tarjeta al sistema.}
    \label{fig:rfid_add}
\end{figure}

\begin{figure}[H]
    \centering
    \includegraphics[width=0.8\linewidth]{Anexos/E7.jpg}
    \caption{Pantalla correspondiente a la opción de eliminar (\texttt{DEL}) una tarjeta previamente registrada.}
    \label{fig:rfid_del}
\end{figure}

En conjunto, los resultados obtenidos demostraron el correcto funcionamiento de los tres módulos desarrollados: el control analógico proporcional del motor, la interfaz de usuario con joystick y matriz RGB, y la cerradura RFID con almacenamiento persistente.  
Cada sistema integró periféricos distintos del microcontrolador, validando su funcionamiento simultáneo y la estabilidad del hardware en condiciones reales de operación.
